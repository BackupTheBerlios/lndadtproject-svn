\chapter{Introduction} \label{chap:Intro}

\begin{quote}
\emph{The most profound technologies are those that disappear. They weave
themselves into the fabric of everyday life until they are indistinguishable
from it} [Mark Weiser]
\end{quote}

\section{Motivation}
Recent advances in Micro-Electro-Mechanical Systems (MEMS)
technologies and wireless communication have made it possible to deploy networks
of sensor nodes called Wireless Sensor Networks (WSNs) that contain a large
number of sensor nodes in several disparate environments. Each individual node is
a multifunctional device characterised by its low cost, low power, and small form
factor. They can communicate untethered across short distances using a variety of
means, including Radio Frequency (RF) communication
\cite{SensorSurveyAkyildiz:2002}.

WSNs are currently used in a wide range of applications including health
monitoring, environment monitoring, data acquisition in dangerous environments,
and target tracking.

However, programming WSNs is currently a complex task that requires an
understanding of the operation of WSNs. This limits its utility in widely
disparate fields by scientists unaware of the technology underlying them. To deal
with this problem, it is necessary to abstract the details of WSN operation
thereby allowing the application programmer to develop WSN applications inspite
of being native the details of embedded system programming. This is
achieved using programming abstractions wherein the WSN application is viewed as a system
and sensor nodes and sensor data are abstracted.

This work underlies the extension of a distributed programming abstraction called
Distributed Abstract Data Types (DADTs) \cite{migliavacca_DADT:2006} that use
high-level programming constructs to abstract interfaces to individual entities
in a distributed network and allow the application programmer to communicate
directly with the the entire distributed network.

Though DADTs appear ideal to the problem domain of abstracting WSN programming,
no work had been performed on examining their suitability, and subsequently
extending them for use in WSN applications. Additionally, a robust routing
mechanism that allows for the selection of a subset of the sensor nodes in a WSN
is required - a requirement that is not met by conventional WSN routing
algorithms.

\section{Contributions}

This work makes the following contributions:

\begin{itemize}
\item It extends the DADT prototype developed in \cite{migliavacca_DADT:2006} to WSNs.
\item It integrates the DADT mechanism in the application layer with a Logical
Neighbourhood \cite{mottola_LNAbstraction} routing mechanism that allows for
the partitioning of the WSN network on the basis of neighbourhoods defined by logical predicates in the network layer.
\item It evaluates the suitability of the developed prototype in simulated
environment as well as in he real-world wireless sensor networks.
\end{itemize}

\section{Outline}
The outline of the thesis is as follows: Chapter \ref{chap:IntroWSN} presents an
introduction to wireless sensor networks, and discusses the protocol stack used
in a typical WSN application and a short survey of routing mechanisms for WSNs.
Chapter \ref{chap:background} discusses the background underlying this work, and
includes brief descriptions of WSN programming models, DADTs, the LN routing mechanism,
and the hardware platform used to evaluate the prototype produced as part of this
work. Chapter \ref{chap:Implementation} describes the details of the
implementation of the prototype as well as a brief description of the structure of the DADT prototype
that was modified during the course of this work. Finally, Chapter \ref{chap:
Evaluation} describes the metrics and experimental methodology used to evaluate this work, the
conclusions thus reached, and possible avenues for future work.








