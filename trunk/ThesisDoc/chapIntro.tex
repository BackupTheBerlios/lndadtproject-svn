\chapter{Introduction} \label{chap:Intro}

\begin{quote}
\emph{The most profound technologies are those that disappear. They weave
themselves into the fabric of everyday life until they are indistinguishable
from it} [Mark Weiser]
\end{quote}

Recent advances in Micro-Electro-Mechanical Systems (MEMS)
technologies and wireless communication have made it possible to deploy networks
of sensor nodes called Wireless Sensor Networks (WSNs) that contain a large
number of sensor nodes in several disparate environments. Each individual node is
a multifunctional device characterised by its low cost, low power, and small form
factor. They can communicate untethered across short distances using a variety of
means, including Radio Frequency (RF) communication
\cite{SensorSurveyAkyildiz:2002}.

WSNs are currently used in a wide range of applications including health
monitoring, environment monitoring, data acquisition in dangerous environments,
and target tracking.

However, programming WSNs is currently a complex task that requires an
understanding of the inner operation of WSNs. This limits its applicability in
widely disparate fields by scientists unaware of the technology underlying WSNs.
To deal with this problem, it is necessary to abstract the details of WSN operation
thereby allowing the application programmer to develop WSN applications in spite
of being an expert embedded system programmer. This is
achieved using programming abstractions wherein the WSN application is viewed as a system
and sensor nodes and sensor data are abstracted.

\section{Motivation}

As was described above, programming for WSNs is an extremely arduous task, which
can be simplified using programming abstraction. Distributed Abstract Data Types
(DADTs) are a distributed programming abstraction that allows for the
distributed state of a system to be manipulated transparently
\cite{migliavacca_DADT:2006}. Though they seem ideal for use in WSNs, DADTs have
not been tested on them.

Conventional WSN routing mechanisms work on the principle of a physical
neighbourhood. However, the data-centric nature of WSNs results in sensor nodes
in different physical neighbourhoods being part of the same logical subset of
nodes. It was to address this that the Logical Neighbourhood (LN)
\cite{mottola_LN:2006} routing mechanism was developed. However, the current
DADT prototype does not use this routing technique.

This work thus attempts to solve the following problems:

\begin{itemize}
  \item Porting the DADT prototype to run on WSNs.
  \item Using an effective routing mechanism to enable distributed programming
  abstractions using DADTs in WSNs.
\end{itemize}

\section{Contributions}

This work makes the following contributions:

\begin{itemize}
\item It extends the DADT prototype developed in \cite{migliavacca_DADT:2006} to WSNs.
\item It integrates the DADT mechanism in the application layer with a Logical
Neighbourhood \cite{mottola_LNAbstraction} routing mechanism that allows for
the partitioning of the WSN network on the basis of neighbourhoods defined by logical predicates in the network layer.
\item It evaluates the suitability of the developed prototype in simulated
environment as well as in real-world wireless sensor networks.
\end{itemize}

\section{Outline}
The outline of the thesis is as follows: Chapter \ref{chap:IntroWSN} presents an
introduction to wireless sensor networks including descriptions of example WSN
applications, provides details about selected hardware platforms and introduces
the protocol stack used in a typical WSN application, coupled with a brief overview of
underlying routing mechanisms for WSNs. Besides, it provides description of WSN
programming models. Chapter \ref{chap:background} discusses the background
underlying this work, and includes brief descriptions of DADTs, the LN routing mechanism, and provides a brief description of the structure of the DADT prototype that was modified during the course of this work. 
Chapter \ref{chap:tools} presents the simluation
environment and the hardware platform used to evaluate the prototype produced as part of this work. 
Chapter \ref{chap:design} describes the high-level design details of the
implementation of the prototype. 
Chapter \ref{chap:implementation} provides implementation details and discusses
the issues concerning testing of the prototype in simulated and in real-wolrd
environment. 
Chapter \ref{chap:evaluation} describes the metrics and experimental methodology
used to evaluate this work.
Finally, Chapter \ref{chap:conclusions} provides the conclusions and possible avenues for future work.








