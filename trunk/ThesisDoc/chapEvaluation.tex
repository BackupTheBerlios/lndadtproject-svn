\chapter{Evaluation} \label{chap: Evaluation}


The metric used to evaluate the performance of the DADT/LN prototype is
the packet processing workload on the application layer. This metric was used to
compare the performance of the DADT/LN prototype against that of the original
DADT prototype used as the basis for this work \cite{migliavacca_DADT:2006}. 

In the original DADT prototype, a request message
was either replied to or discarded by the application layer entity on the basis of the
evaluation of an abstract tree representing the specified scope of the operation. In
the implementation presented in this work, the integration with the LN approach
results in unsuitable request messages being discarded on the basis of predicate
evaluation in the network layer.

//add the graph here

A series of simulations were run using the JiST/SWANS simulator to determine
the number of request messages discarded in the network layer, and the number
passed on to the application layer by the LN predicate matching algorithm. 
It was found that for every distributed operation request that was limited in
scope to a set of ADT instances present only in a subset of the sensor nodes in
the network, the number of packets forwarded to the DADT/LN prototype's
application layer for processing was lower than the corresponding number in the
original DADT prototype. In the worst case scenario where every sensor node in
the network has at least one ADT instance that falls within the scope of the
operation, the application layer traffic on both prototypes is the same.

