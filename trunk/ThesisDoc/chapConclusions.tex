\chapter{Evaluation, Conclusions, and Future Work} \label{chap: Evaluation}
 \section{Evaluation}

The performance of the DADT/LN prototype was evaluated using the following
metrics: 
\begin{itemize}
  \item Packet processing workload on the application layer.
  \item Ease of implementation. 
\end{itemize}

The first metric was used to compare the performance of the DADT/LN prototype
against that of the original DADT prototype used as the basis for this work
\cite{migliavacca_DADT:2006}. In the original DADT prototype, a request message
was replied to or discarded in the application layer on the basis of the
evaluation of an abstract tree representing specified scope of the operation.
In the implementation presented in this work, the integration with the LN approach results in unsuitable request messages being discarded on
the basis of predicate evaluation in the network layer. 

A series of simulations were run using the JiST/SWANS simulations to determine
the number of request messages discarded at the network layer and the number
passed on to the application layer by the LN predicate matching algorithm. The
sum provides the total number of packets processed in the application layer of
the original DADT prototype.

It was found that the number of messages processed in the application layer was
lower in the implementation produced as part of this work.

%HOW CAN WE SAY LN PREDICATE EVAL IS MORE EFFICIENT THAN EXPRESSION TREE?

%
%The second metric used the number of lines of code required to implement a
%simple distributed averaging application. While not an ideal metric, it
%quantifies the difference in the order of magnitude of coding effort required
%on the part of the application programmer to produce WSN applications with and
%without the use of programming abstractions. To this end, a simple WSN
%application was written on the Sun SPOTs to calculate the distributed average.

%The difference in the number of lines was **.

\section{Conclusions}

%HERE MORE BLAH BLAH

\section{Future Work}

This section presents a list of possible extensions to the work implemented as
part of this thesis. These include:

\begin{itemize}
  \item \emph{Support for DADT selection operators:} The current prototype
  supports the selection of all ADT instances that match a defined DADT Data
  view, but does not enable the selection of a subset of the aforementioned
  collection of ADT instances. This arises from the limitations of the current
  LN implementation.
  \item \emph{Extending support for Space DADTs:} Currently, the prototype
  provides a limited support for the notion of space. Therefore, a possible
  avenue for future work could include the full support for Space DADTs
  provided by the prototype.
  \item \emph{Extending the prototype for networks of heterogenous nodes:}
  The current prototype, by virtue of it being implemented in Java, cannot be
  used on a wide variety of different nodes. 
\end{itemize}
