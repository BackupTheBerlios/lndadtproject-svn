\chapter{Conclusions, and Future Work} \label{chap:conclusions}
 \section{Conclusions}

During the course of this project, it was proven that the concept of DADTs can be
applied to real world WSNs. The proof-of-concept prototype implemented indicates
the potential of programming abstractions in helping reduce the effort involved
in developing applications for WSNs.
In addition, upon integrating the innovative LN routing mechanism
with the DADT implementation on the application layer, there was found to be a
significant reduction in the amount of processing performed in the application
layer. 

Evaluation of the prototype showed that the DADT/LN prototype was scalable and
could handle increasingly complex and concurrent requests with only small
reductions in performance.
 
\section{Future Work}

This section presents a list of possible extensions to the work implemented as
part of this thesis. These include:

\begin{itemize}
  \item \emph{Support for DADT selection operators:} The current prototype
  supports the selection of all ADT instances that match a defined DADT Data
  view, but does not enable the selection of a subset of the aforementioned
  collection of ADT instances. This arises from the limitations of the current
  LN implementation.
  \item \emph{Extending support for Space DADTs:} Currently, the prototype
  provides a limited support for the notion of space. Therefore, a possible
  avenue for future work could include the full support for Space DADTs
  provided by the prototype.
  \item \emph{Extending the prototype for networks of heterogenous nodes:}
  The current prototype, by virtue of it being implemented in Java, cannot be
  used on a wide variety of different nodes. 
\end{itemize}
