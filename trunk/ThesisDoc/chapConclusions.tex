\chapter{Evaluation, Conclusions, and Future Work} \label{chap: Evaluation}
 \section{Evaluation}

The metric used to evaluate the performance of the DADT/LN prototype is
the packet processing workload on the application layer. This metric was used to
compare the performance of the DADT/LN prototype against that of the original
DADT prototype used as the basis for this work \cite{migliavacca_DADT:2006}. 

In the original DADT prototype, a request message
was either replied to or discarded by the application layer entity on the basis of the
evaluation of an abstract tree representing the specified scope of the operation. In
the implementation presented in this work, the integration with the LN approach
results in unsuitable request messages being discarded on the basis of predicate
evaluation in the network layer.

//add the graph here

A series of simulations were run using the JiST/SWANS simulator to determine
the number of request messages discarded in the network layer, and the number
passed on to the application layer by the LN predicate matching algorithm. 
It was found that for every distributed operation request that was limited in
scope to a set of ADT instances present only in a subset of the sensor nodes in
the network, the number of packets forwarded to the DADT/LN prototype's
application layer for processing was lower than the corresponding number in the
original DADT prototype. In the worst case scenario where every sensor node in
the network has at least one ADT instance that falls within the scope of the
operation, the application layer traffic on both prototypes is the same.

\section{Conclusions}

During the course of this project, it was proven that the concept of DADTs can be
applied to real world WSNs. The proof-of-concept prototype implemented indicates
the potential of programming abstractions in helping reduce the effort involved
in developing applications for WSNs.
In addition, upon integrating the innovative LN routing mechanism
with the DADT implementation on the application layer, there was found to be a
significant reduction in the amount of processing performed in the application
layer. 
 
\section{Future Work}

This section presents a list of possible extensions to the work implemented as
part of this thesis. These include:

\begin{itemize}
  \item \emph{Support for DADT selection operators:} The current prototype
  supports the selection of all ADT instances that match a defined DADT Data
  view, but does not enable the selection of a subset of the aforementioned
  collection of ADT instances. This arises from the limitations of the current
  LN implementation.
  \item \emph{Extending support for Space DADTs:} Currently, the prototype
  provides a limited support for the notion of space. Therefore, a possible
  avenue for future work could include the full support for Space DADTs
  provided by the prototype.
  \item \emph{Extending the prototype for networks of heterogenous nodes:}
  The current prototype, by virtue of it being implemented in Java, cannot be
  used on a wide variety of different nodes. 
\end{itemize}
